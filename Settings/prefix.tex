% \documentclass[rusmathsym, eqnumwithinsec, amspack, hyperref]{bomgost}
% \input{letters.def}

% Закомментируйте это, если hyperref не нужен. 
% Изменение уровня \subparagraph (приложения) в закладках pdf документа. 
% Это нужно для сохранения правильной иерархии закладок в pdf документе.
\makeatletter%
\renewcommand{\toclevel@subparagraph}{2}%
\makeatother% 

% Для вставки программного кода.
\usepackage{listings}

% "Умная" запятая: \(0,2\) - число, \(0, 2\) - перечисление.
\usepackage{icomma}
\usepackage{float}
\usepackage{pgfplots}
\usepackage{pgfplotstable}
\usepackage{longtable}
\usepackage{cleveref}

\usepackage{lastpage}
\usepackage{refcount}

% Графика
\usepackage{graphicx} % подключение графики

\usepackage{multirow} % объединение ячеек таблиц по вертикали
\usepackage{tabularx} % автоматический подбор ширины столбцов

\pgfplotsset{compat=newest}

\pgfplotstableset{set thousands separator={}, precision=3, use comma, col sep=comma, header=true,
every head row/.style={before row=\hline, after row=\hline},
every even row/.style={after row=\hline},
every odd row/.style={after row=\hline},
every column/.style={column type/.add={|}{}},
every last column/.style={column type/.add={}{|}},
columns/text/.style={string type},
}


% Гиперссылки
% \usepackage{hyperref}

% Прочее
\usepackage{color} % работа с цветом
\usepackage{dashrule}
\usepackage{lscape}

\usepackage{afterpage} % вставка материала после текущей страницы
\usepackage[font={normal}]{caption} % настройка подписей к рисункам и таблицам

% Использование полужирного начертания для векторов
\let\vec=\mathbf

% Таблицы
\usepackage{array} % расширенные возможности для работы с таблицами

\usepackage{dcolumn} % выравнивание чисел по разделителям

\usepackage[backend=biber, style=gost-numeric, sorting=none]{biblatex}

\addbibresource{Bibliography/bibliography.bib}

%New colors defined below
\definecolor{codegreen}{rgb}{0,0.6,0}
\definecolor{codegray}{rgb}{0.5,0.5,0.5}
\definecolor{codepurple}{rgb}{0.58,0,0.82}
\definecolor{backcolour}{rgb}{0.96,0.96,0.96}

%Code listing style named "mystyle"
\lstdefinestyle{mystyle}{
    numbers=left,   % С какой стороны нумеровать
    stepnumber = 3, % Шаг нумерации 
    numbersep=5pt,                      
    numberstyle=\color{codegray},       % Стиль который будет использоваться для нумерации строк
    backgroundcolor=\color{backcolour}, % Цвет фона
    commentstyle=\color{codegreen},     % Стиль комментариев
    keywordstyle=\color{magenta},       % Стиль ключевых слов
    stringstyle=\color{codepurple},     % Стиль литералов
    basicstyle=\ttfamily\footnotesize,  % Используемый шрифт
    breaklines=true,        % Автоматический перенос строк
    breakatwhitespace=false, % Переносить строки по словам
    tabsize=2,                          % Tab - 2 пробела
    showtabs=false,                     % Показывать отступы
    showspaces=false,
    showstringspaces=false,             % Показывать пробелы
    keepspaces=true
}


\lstset{style=mystyle}

\pretocmd{\cite}{\stepcounter{totreferences}}{}{}