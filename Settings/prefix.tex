% \documentclass[rusmathsym, eqnumwithinsec, amspack, hyperref]{bomgost}
% \DeclareTextSymbol{\CYRA}\UnicodeEncodingName{"0410}        % А
\DeclareTextSymbol{\cyra}\UnicodeEncodingName{"0430}        % а
\DeclareTextSymbol{\CYRB}\UnicodeEncodingName{"0411}        % Б
\DeclareTextSymbol{\cyrb}\UnicodeEncodingName{"0431}        % б
\DeclareTextSymbol{\CYRV}\UnicodeEncodingName{"0412}        % В 
\DeclareTextSymbol{\cyrv}\UnicodeEncodingName{"0432}        % в
\DeclareTextSymbol{\CYRG}\UnicodeEncodingName{"0413}        % Г
\DeclareTextSymbol{\cyrg}\UnicodeEncodingName{"0433}        % г
\DeclareTextSymbol{\CYRD}\UnicodeEncodingName{"0414}        % Д
\DeclareTextSymbol{\cyrd}\UnicodeEncodingName{"0434}        % д
\DeclareTextSymbol{\CYRE}\UnicodeEncodingName{"0415}        % Е 
\DeclareTextSymbol{\cyre}\UnicodeEncodingName{"0435}        % е
\DeclareTextSymbol{\CYRZH}\UnicodeEncodingName{"0416}       % Ж 
\DeclareTextSymbol{\cyrzh}\UnicodeEncodingName{"0436}       % ж
\DeclareTextSymbol{\CYRZ}\UnicodeEncodingName{"0417}        % З
\DeclareTextSymbol{\cyrz}\UnicodeEncodingName{"0437}        % з
\DeclareTextSymbol{\CYRI}\UnicodeEncodingName{"0418}        % И
\DeclareTextSymbol{\cyri}\UnicodeEncodingName{"0438}        % и
\DeclareTextSymbol{\CYRISHRT}\UnicodeEncodingName{"0419}    % Й
\DeclareTextSymbol{\cyrishrt}\UnicodeEncodingName{"0439}    % й
\DeclareTextSymbol{\CYRK}\UnicodeEncodingName{"041A}        % К
\DeclareTextSymbol{\cyrk}\UnicodeEncodingName{"043A}        % к
\DeclareTextSymbol{\CYRL}\UnicodeEncodingName{"041B}        % Л
\DeclareTextSymbol{\cyrl}\UnicodeEncodingName{"043B}        % л 
\DeclareTextSymbol{\CYRM}\UnicodeEncodingName{"041C}        % М
\DeclareTextSymbol{\cyrm}\UnicodeEncodingName{"043C}        % м
\DeclareTextSymbol{\CYRN}\UnicodeEncodingName{"041D}        % Н
\DeclareTextSymbol{\cyrn}\UnicodeEncodingName{"043D}        % н
\DeclareTextSymbol{\CYRO}\UnicodeEncodingName{"041E}        % О
\DeclareTextSymbol{\cyro}\UnicodeEncodingName{"043E}        % о
\DeclareTextSymbol{\CYRP}\UnicodeEncodingName{"041F}        % П
\DeclareTextSymbol{\cyrp}\UnicodeEncodingName{"043F}        % п
\DeclareTextSymbol{\CYRR}\UnicodeEncodingName{"0420}        % Р
\DeclareTextSymbol{\cyrr}\UnicodeEncodingName{"0440}        % р
\DeclareTextSymbol{\CYRS}\UnicodeEncodingName{"0421}        % С
\DeclareTextSymbol{\cyrs}\UnicodeEncodingName{"0441}        % с
\DeclareTextSymbol{\CYRT}\UnicodeEncodingName{"0422}        % Т
\DeclareTextSymbol{\cyrt}\UnicodeEncodingName{"0442}        % т
\DeclareTextSymbol{\CYRU}\UnicodeEncodingName{"0423}        % У
\DeclareTextSymbol{\cyru}\UnicodeEncodingName{"0443}        % у
\DeclareTextSymbol{\CYRF}\UnicodeEncodingName{"0424}        % Ф
\DeclareTextSymbol{\cyrf}\UnicodeEncodingName{"0444}        % ф
\DeclareTextSymbol{\CYRH}\UnicodeEncodingName{"0425}        % Х
\DeclareTextSymbol{\cyrh}\UnicodeEncodingName{"0445}        % х
\DeclareTextSymbol{\CYRC}\UnicodeEncodingName{"0426}        % Ц
\DeclareTextSymbol{\cyrc}\UnicodeEncodingName{"0446}        % ц
\DeclareTextSymbol{\CYRCH}\UnicodeEncodingName{"0427}       % Ч
\DeclareTextSymbol{\cyrch}\UnicodeEncodingName{"0447}       % ч
\DeclareTextSymbol{\CYRSH}\UnicodeEncodingName{"0428}       % Ш
\DeclareTextSymbol{\cyrsh}\UnicodeEncodingName{"0448}       % ш
\DeclareTextSymbol{\CYRSHCH}\UnicodeEncodingName{"0429}     % Щ
\DeclareTextSymbol{\cyrshch}\UnicodeEncodingName{"0449}     % щ
\DeclareTextSymbol{\CYRHRDSN}\UnicodeEncodingName{"042A}    % Ъ
\DeclareTextSymbol{\cyrhrdsn}\UnicodeEncodingName{"044A}    % ъ
\DeclareTextSymbol{\CYRERY}\UnicodeEncodingName{"042B}      % Ы
\DeclareTextSymbol{\cyrery}\UnicodeEncodingName{"044B}      % ы
\DeclareTextSymbol{\CYRSFTSN}\UnicodeEncodingName{"042C}    % Ь
\DeclareTextSymbol{\cyrsftsn}\UnicodeEncodingName{"044C}    % ь
\DeclareTextSymbol{\CYREREV}\UnicodeEncodingName{"042D}     % Э
\DeclareTextSymbol{\cyrerev}\UnicodeEncodingName{"044D}     % э
\DeclareTextSymbol{\CYRYU}\UnicodeEncodingName{"042E}       % Ю
\DeclareTextSymbol{\cyryu}\UnicodeEncodingName{"044E}       % ю
\DeclareTextSymbol{\CYRYA}\UnicodeEncodingName{"042F}       % Я
\DeclareTextSymbol{\cyrya}\UnicodeEncodingName{"044F}       % я

% Закомментируйте это, если hyperref не нужен. 
% Изменение уровня \subparagraph (приложения) в закладках pdf документа. 
% Это нужно для сохранения правильной иерархии закладок в pdf документе.
\makeatletter%
\renewcommand{\toclevel@subparagraph}{2}%
\makeatother% 

% Для вставки программного кода.
\usepackage{listings}

% "Умная" запятая: \(0,2\) - число, \(0, 2\) - перечисление.
\usepackage{icomma}
\usepackage{float}
\usepackage{pgfplots}
\usepackage{pgfplotstable}
\usepackage{longtable}
\usepackage{cleveref}

\usepackage{lastpage}
\usepackage{refcount}

% Графика
\usepackage{graphicx} % подключение графики

\usepackage{multirow} % объединение ячеек таблиц по вертикали
\usepackage{tabularx} % автоматический подбор ширины столбцов

\pgfplotsset{compat=newest}

\pgfplotstableset{set thousands separator={}, precision=3, use comma, col sep=comma, header=true,
every head row/.style={before row=\hline, after row=\hline},
every even row/.style={after row=\hline},
every odd row/.style={after row=\hline},
every column/.style={column type/.add={|}{}},
every last column/.style={column type/.add={}{|}},
columns/text/.style={string type},
}


% Гиперссылки
% \usepackage{hyperref}

% Прочее
\usepackage{color} % работа с цветом
\usepackage{dashrule}
\usepackage{lscape}

\usepackage{afterpage} % вставка материала после текущей страницы
\usepackage[font={normal}]{caption} % настройка подписей к рисункам и таблицам

% Использование полужирного начертания для векторов
\let\vec=\mathbf

% Таблицы
\usepackage{array} % расширенные возможности для работы с таблицами

\usepackage{dcolumn} % выравнивание чисел по разделителям

\usepackage[backend=biber, style=gost-numeric, sorting=none]{biblatex}

\addbibresource{Bibliography/bibliography.bib}

%New colors defined below
\definecolor{codegreen}{rgb}{0,0.6,0}
\definecolor{codegray}{rgb}{0.5,0.5,0.5}
\definecolor{codepurple}{rgb}{0.58,0,0.82}
\definecolor{backcolour}{rgb}{0.96,0.96,0.96}

%Code listing style named "mystyle"
\lstdefinestyle{mystyle}{
    numbers=left,   % С какой стороны нумеровать
    stepnumber = 3, % Шаг нумерации 
    numbersep=5pt,                      
    numberstyle=\color{codegray},       % Стиль который будет использоваться для нумерации строк
    backgroundcolor=\color{backcolour}, % Цвет фона
    commentstyle=\color{codegreen},     % Стиль комментариев
    keywordstyle=\color{magenta},       % Стиль ключевых слов
    stringstyle=\color{codepurple},     % Стиль литералов
    basicstyle=\ttfamily\footnotesize,  % Используемый шрифт
    breaklines=true,        % Автоматический перенос строк
    breakatwhitespace=false, % Переносить строки по словам
    tabsize=2,                          % Tab - 2 пробела
    showtabs=false,                     % Показывать отступы
    showspaces=false,
    showstringspaces=false,             % Показывать пробелы
    keepspaces=true
}


\lstset{style=mystyle}

\pretocmd{\cite}{\stepcounter{totreferences}}{}{}