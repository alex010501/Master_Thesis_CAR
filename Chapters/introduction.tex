\newpage

\begin{abstract}

Пояснительная записка к научно-исследовательской работе содержит \printtotpage \printtotfig \printtottab \printtotref \printtotapp[.]

Ключевые слова: промышленная робототехника, нейронные сети математическое моделирование, компьютерное моделирование.

Работа рассматривает применение нейронных сетей в алгоритме управления промышленным манипуляционным роботом для расчёта оптимальной траектории движения. Поставлены цели и задачи исследования, проведён анализ предметной области и научной литературы, проведён обзор средств компьютерного моделирования, сформулированы требования к разрабатываемому программному обеспечению для работы с моделью робота-манипулятора.
\end{abstract}

\tableofcontents


\section*{ОБОЗНАЧЕНИЯ И СОКРАЩЕНИЯ}
ПМР -- промышленные манипуляционные роботы

НС -- нейронные сети

РТК -- робототехнический комплекс

ПЗК -- прямая задача кинематики

ОЗК -- обратная задача кинематики

ОЗД -- обратная задача динамики

TCP -- Tool center point

BFGS -- Broyden-Fletcher-Goldfarb-Shanno

\section*{ВВЕДЕНИЕ}

Согласно словарю Вебстера, \textit{робот -- автоматическое устройство,\hspace{0,2cm} выполняющее функции, обычно приписываемые человеку}. Более точно промышленный робот можно охарактеризовать как: перепрограммируемый многофункциональный манипулятор, предназначенный для осуществления различных заранее заданных перемещений материалов, деталей, инструментов или специальных приспособлений с целью выполнения различных работ. Итак, робот представляет собой перепрограммируемый универсальный манипулятор, снабженный внешними датчиками и способный выполнять различные производственные задачи. Это определение предполагает наличие у робота интеллекта, обусловленного заложенными в компьютер алгоритмами систем управления и очувствления.
Следовательно, промышленный робот представляет собой универсальный, оснащенный компьютером манипулятор, состоящий из нескольких твердых звеньев, последовательно соединенных вращательными или поступательными сочленениями. \cite{Fu_Robot_Book}

Промышленные манипуляционные роботы (ПМР) способны выполнять различные операции по перемещению, сборке, сварке, покраске и другим видам обработки объектов. ПМР широко используются в современном производстве, так как они повышают производительность, качество и безопасность труда. Однако, для эффективного использования ПМР необходимо решать ряд сложных задач, связанных с планированием и управлением их движения.

Одной из таких задач является расчёт траекторий движения ПМР, то есть определение последовательности положений и ориентаций звеньев робота в пространстве, которые обеспечивают выполнение заданной операции. Эта задача имеет большое практическое значение, так как от качества траектории зависят скорость, точность и безопасность движения робота. Традиционные методы расчёта траекторий основаны на аналитических или численных алгоритмах, которые требуют знания геометрии, кинематики и динамики робота, а также учёта механических и программных ограничений. Эти методы имеют ряд недостатков, таких как высокая вычислительная сложность, низкая универсальность и адаптивность, чувствительность к ошибкам измерения и настройки параметров.

В связи с этим, в последнее время активно развиваются альтернативные подходы к расчёту траекторий движения ПМР, основанные на использовании нейронных сетей. Нейронные сети (НС) -- искусственные, многослойные высокопараллельные логические структуры, составленные из формальных нейронов.\cite{Big_Rus_Book}
НС способны обучаться на данных и аппроксимировать сложные нелинейные зависимости. Преимущества НС заключаются в том, что они не требуют явного знания модели робота и ограничений, а также могут адаптироваться к изменяющимся условиям и обобщать результаты на новые ситуации. Существует множество видов НС, различающихся по архитектуре, функциям активации, алгоритмам обучения и применению. В зависимости от поставленной задачи, могут использоваться разные типы НС, такие как многослойные, рекуррентные, свёрточные, глубокие нейронные сети.\cite{NW_Class_Art}

\newpage
Целью данной работы является \textbf{разработка методов применения нейронных сетей для расчёта траекторий движения промышленного манипуляционного робота}.

Для достижения этой цели были поставлены следующие задачи:
\begin{enumerate}
\item Изучить основные принципы и методы компьютерного моделирования кинематики и динамики ПМР, а также существующие программные средства для этой цели.
\item Разработать программное обеспечение для компьютерного моделирования работы ПМР по следующему алгоритму:
    \begin{enumerate}[label=\asbuk*)]
    \item спроектировать архитектуру и функциональность системы, учитывая требования к надёжности, производительности, удобству и гибкости;
    \item выбрать подходящие технологии и инструменты для реализации системы, такие как языки программирования, библиотеки, фреймворки и среды разработки;
    \item разработать модули системы, отвечающие за ввод исходных данных, расчёт кинематических и динамических параметров, визуализацию результатов и взаимодействие с пользователем;
    \item интегрировать модули системы в единое приложение и провести его отладку и тестирование на различных примерах движения ПМР;
\end{enumerate}
\item Провести обзор существующих методов расчёта траекторий движения ПМР, основанных на НС, и выделить их достоинства и недостатки.
\item Выбрать подходящий тип НС и разработать алгоритм её обучения на основе синтетических и реальных данных о движении ПМР.
\item Реализовать программную модель НС и провести её тестирование на различных сценариях движения ПМР.
\item Сравнить полученные результаты с традиционными методами расчёта траекторий и оценить эффективность применения НС.
\end{enumerate}